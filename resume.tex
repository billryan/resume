% !TEX program = xelatex

\documentclass{resume}
%\usepackage{zh_CN-Adobefonts_external} % Simplified Chinese Support using external fonts (./fonts/zh_CN-Adobe/)
%\usepackage{zh_CN-Adobefonts_internal} % Simplified Chinese Support using system fonts

\begin{document}
\pagenumbering{gobble} % suppress displaying page number

\name{Yi Xu}

\basicInfo{
  \email{xuyi18@mails.tsinghua.edu.cn} \textperiodcentered\ 
  \phone{(+86) 138-1397-8515}}
  %\linkedin[billryan8]{https://www.linkedin.com/in/billryan8}}

\section{Education}
\datedsubsection{\textbf{Tsinghua University}, Beijing, China}{2018 -- Present}
\textit{Undergraduate student} in Department of Computer Science and Technology (CST)

\section{Experience}
\datedsubsection{\textbf{Taichi Graphics Technology, Inc.}}{Beijing, China}
\datedsubsection{Research and Development Intern}{2021.04 -- Present}
\begin{itemize}
  \item Propose and deploy Meshtaichi, with 1.4x to 6x speedup over state-of-the-art systems.
  %\item Implement a real-time invertible FEM.
  \item Implement large-scale real-time simulations including cloth with self-collision and invertible FEM.
\end{itemize}

\section{Projects}

\subsection{\textbf{MeshTaichi: A Compiler for Efficient Mesh-based Operations}}
\begin{itemize}
  %\item Publish Meshtaichi in SIGGRAPH Asia 2022 as joint first author.
        %Meshtaichi provides a user-friendly syntax and achieves a consistant speedup ranging from 1.4x to 6x over compared to state-of-the-art systems.
  \item Further Exploit data locality and utilize on-chip memory by partitioning meshes into smaller patches, where our compiler takes over the caching scheme and reduces attribute IO overhead.
  \item Conduct a through evaluation of our system by comparing with other languages, including CUDA and Taichi, in several mesh-based applications.
  %several mesh-based applications as comparison experiments in different systems, including Taichi and CUDA.
  \item Propose a new mesh partitioning strategy, and improves the performance by 20\% for tetrahedron mesh by reducing ribbon elements shared by multiple patches.
\end{itemize}

\subsection{\textbf{Real-time physically-based simulations on mesh}}
\begin{itemize}
  \item Implement a real-time cloth simultion with the repulsion method for cloth self collisions (Wang 2021), quadratic bending model and SDF fields.
        Runs at 50 fps with 30k vertices on RTX 2060.
  \item Implement a invertible finite elements based on Projective Dynamics as a Meshtaichi testcase, able to handle extremely large deformations.
        Runs at 34 fps with 200k vertices on RTX 3090.

\end{itemize}

\subsection{\textbf{GAMES 201 course project: MPM}}
\begin{itemize}
  \item Reproduce CPIC (Hu 2018), supporting two-way coupling with rigid bodies and material cutting.
  \item Couple Neo-hookean FEM lagrangian forces with MPM frame.
\end{itemize}

\subsection{\textbf{Physically-based rendering}}
\begin{itemize}
  \item Implement Monte Carlo Path Tracing and SPPM, with k-d tree acceleration.
\end{itemize}

\section{Awards}
\begin{itemize}
  \item Gold Prize, National Olympiad in Informatics (NOI)\hfill 2016
  \item Scholarship of Academic Excellence (top 20\%)\hfill 2021
\end{itemize}

\section{Publications}
\begin{itemize}
  \item Chang Yu*, \textbf{Yi Xu}* (*joint first authors), Ye Kuang, Yuanming Hu, Tiantian Liu   
  ``MeshTaichi: A Compiler for Efficient Mesh-based Operations'', 
  \textbf{ACM Transactions on Graphics [Proceedings of SIGGRAPH Asia], 2022}.

	
  \item Tianhui Shi, Mingshu Zhai, \textbf{Yi Xu}, Jidong Zhai, 
  ``GraphPi: High Performance Graph Pattern Matching through Effective Redundancy Elimination'', 
  \textbf{SC'20}.
  % \begin{itemize}
	% \item Implement a multi-machine version of GraphPi, scaling up to 1024 computing nodes.
  %(24576 processor cores)
  % \end{itemize}
\end{itemize}

\section{Skills}
\textbf{Programming Languages:} \small C/C++, Python, CUDA %(ranked by proficiency)

\textbf{Languages:} \small Mandarin Chinese (Native speaker), English (TOEFL 107), Japanese (JLPT N1)

%% Reference
%\newpage
%\bibliographystyle{IEEETran}
%\bibliography{mycite}
\end{document}
